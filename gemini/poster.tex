% Gemini theme
% https://github.com/anishathalye/gemini

\documentclass[final]{beamer}

% ====================
% Packages
% ====================

\usepackage[T1]{fontenc}
\usepackage{lmodern}
% \usepackage[size=custom,width=120,height=72,scale=1.0]{beamerposter}
\usepackage[size=B0]{beamerposter}
\usetheme{gemini}
\usecolortheme{gemini}
\usepackage{graphicx}
\usepackage{booktabs}
\usepackage{tikz}
\usepackage{multicol}

\usepackage{tikz-3dplot}
\usetikzlibrary{3d}

\usepackage{pgfplots}
\pgfplotsset{compat=1.14}
\usepackage{anyfontsize}

% ====================
% Lengths
% ====================

% If you have N columns, choose \sepwidth and \colwidth such that
% (N+1)*\sepwidth + N*\colwidth = \paperwidth
\newlength{\sepwidth}
\newlength{\colwidth}
\setlength{\sepwidth}{0.025\paperwidth}
\setlength{\colwidth}{0.3\paperwidth}

\newcommand{\separatorcolumn}{\begin{column}{\sepwidth}\end{column}}

% ====================
% Title
% ====================

\title{Understanding The Rubik's Cube: A Group Theoretic Approach}

% \author{Tony X \inst{1} \and Pesara A \inst{1} \and Zeyn J \inst{1}}
\author{Tony X \and Pesara A  \and Zeyn J}

%\institute[shortinst]{\inst{1} Some Institute \samelineand \inst{2} Another Institute}
% \institute[shortinst]{\inst{1} McMaster University}
% ====================
% Footer (optional)
% ====================

%\footercontent{
%  \href{https://www.example.com}{https://www.example.com} \hfill
%  ABC Conference 2025, New York --- XYZ-1234 \hfill
%  \href{mailto:alyssa.p.hacker@example.com}{alyssa.p.hacker@example.com}}
% (can be left out to remove footer)

% ====================
% Logo (optional)
% ====================

% use this to include logos on the left and/or right side of the header:
% \logoright{\includegraphics[height=7cm]{logo1.pdf}}
% \logoleft{\includegraphics[height=7cm]{logo2.pdf}}

% ====================
% Body
% ====================

\begin{document}
\pgfmathsetmacro\radius{0.1}
\newcommand{\frontcolor}{red}
\newcommand{\sidecolor}{blue}
\begin{frame}[t]
\begin{columns}[t]
\separatorcolumn

\begin{column}{\colwidth}

  % \begin{block}{Abstract}

   % Some block contents, followed by a diagram, followed by a dummy paragraph.

%%    \begin{figure}
%%      \centering
%%      \begin{tikzpicture}[scale=6]
%%        \draw[step=0.25cm,color=gray] (-1,-1) grid (1,1);
%%        \draw (1,0) -- (0.2,0.2) -- (0,1) -- (-0.2,0.2) -- (-1,0)
%%          -- (-0.2,-0.2) -- (0,-1) -- (0.2,-0.2) -- cycle;
%%      \end{tikzpicture}
%%      \caption{A figure caption.}
  %%  \end{figure}

    % The rubiks cube is perhaps the world's most famous puzzle. 
    % The first rigorous 
    % mathematical analysis of a Rubik's Cube was done by British-American Mathemetician 
    % David Singmaster, though solutions were known to a number of 
    % mathematicians at the time (such as J.H. Conway and Roger Penrose).  
    % In this project, we attempt to understand the inner workings of the Rubik's cube via group theory and abstract algebra. 

              % \begin{figure}
              %   \centering
              %         \begin{tikzpicture}[z={(3.85mm,3.85mm)}]
              %             \coordinate (O) at (0,0,0);
              %             \coordinate (A) at (1,0,0);
              %             \coordinate (B) at (2,0,0);
              %             \coordinate (C) at (3,0,0);
              %             \coordinate (D) at (0,1,0);
              %             \coordinate (G) at (1,1,0);
              %             \coordinate (L) at (2,1,0);
              %             \coordinate (N) at (3,1,0);
              %             \coordinate (E) at (0,2,0);
              %             \coordinate (J) at (1,2,0);
              %             \coordinate (H) at (2,2,0);
              %             \coordinate (P) at (3,2,0);
              %             \coordinate (F) at (0,3,0);
              %             \coordinate (K) at (1,3,0);
              %             \coordinate (M) at (2,3,0);
              %             \coordinate (I) at (3,3,0);
              %             \coordinate (Q) at (3,0,1);
              %             \coordinate (R) at (3,0,2);
              %             \coordinate (S) at (3,0,3);
              %             \coordinate (T) at (3,1,1);
              %             \coordinate (U) at (3,1,2);
              %             \coordinate (V) at (3,1,3);
              %             \coordinate (W) at (3,2,1);
              %             \coordinate (X) at (3,2,2);
              %             \coordinate (Y) at (3,2,3);
              %             \coordinate (Z) at (3,3,1);
              %             \coordinate (AA) at (3,3,2);
              %             \coordinate (BB) at (3,3,3);
              %             \coordinate (CC) at (0,3,1);
              %             \coordinate (DD) at (1,3,1);
              %             \coordinate (EE) at (2,3,1);
              %             \coordinate (FF) at (0,3,2);
              %             \coordinate (GG) at (1,3,2);
              %             \coordinate (HH) at (2,3,2);
              %             \coordinate (II) at (0,3,3);
              %             \coordinate (JJ) at (1,3,3);
              %             \coordinate (KK) at (2,3,3);
                          
              %             \draw[black,fill=blue!80] (O) -- (A) -- (G) -- (D) -- cycle;
              %             \draw[black,fill=blue!80] (A) -- (B) -- (L) -- (G) -- cycle;
              %             \draw[black,fill=blue!80] (B) -- (C) -- (N) -- (L) -- cycle;
              %             \draw[black,fill=blue!80] (D) -- (G) -- (J) -- (E) -- cycle;
              %             \draw[black,fill=blue!80] (G) -- (L) -- (H) -- (J) -- cycle;
              %             \draw[black,fill=blue!80] (L) -- (N) -- (P) -- (H) -- cycle;
              %             \draw[black,fill=blue!80] (E) -- (J) -- (K) -- (F) -- cycle;
              %             \draw[black,fill=blue!80] (J) -- (H) -- (M) -- (K) -- cycle;
              %             \draw[black,fill=blue!80] (H) -- (P) -- (I) -- (M) -- cycle;
                          
              %             \draw[black,fill=white!80] (C) -- (Q) -- (T) -- (N) -- cycle;
              %             \draw[black,fill=white!80] (Q) -- (R) -- (U) -- (T) -- cycle;
              %             \draw[black,fill=white!80] (R) -- (S) -- (V) -- (U) -- cycle;
              %             \draw[black,fill=white!80] (N) -- (T) -- (W) -- (P) -- cycle;
              %             \draw[black,fill=white!80] (T) -- (U) -- (X) -- (W) -- cycle;
              %             \draw[black,fill=white!80] (U) -- (V) -- (Y) -- (X) -- cycle;
              %             \draw[black,fill=white!80] (P) -- (W) -- (Z) -- (I) -- cycle;
              %             \draw[black,fill=white!80] (W) -- (X) -- (AA) -- (Z) -- cycle;
              %             \draw[black,fill=white!80] (X) -- (Y) -- (BB) -- (AA) -- cycle;
                          
              %             \draw[black,fill=red!80] (F) -- (K) -- (DD) -- (CC) -- cycle;
              %             \draw[black,fill=red!80] (K) -- (M) -- (EE) -- (DD) -- cycle;
              %             \draw[black,fill=red!80] (M) -- (I) -- (Z) -- (EE) -- cycle;
              %             \draw[black,fill=red!80] (CC) -- (DD) -- (GG) -- (FF) -- cycle;
              %             \draw[black,fill=red!80] (DD) -- (EE) -- (HH) -- (GG) -- cycle;
              %             \draw[black,fill=red!80] (EE) -- (Z) -- (AA) -- (HH) -- cycle;
              %             \draw[black,fill=red!80] (FF) -- (GG) -- (JJ) -- (II) -- cycle;
              %             \draw[black,fill=red!80] (GG) -- (HH) -- (KK) -- (JJ) -- cycle;
              %             \draw[black,fill=red!80] (HH) -- (AA) -- (BB) -- (KK) -- cycle;
              %             \end{tikzpicture}
              %             \caption{A 3x3x3 Rubik's Cube}
              %           \end{figure}

  % \end{block}

  % \begin{block}{Modelling the Cube}


  % \end{block}

  \begin{alertblock}{Useful Definitions}


  \heading{Semi-direct products:}
     Let $G$ be a group with identity element $e$, a subgroup $H$ and a normal subgroup $N \triangleleft G$ such that $N \cap H = \{e\}$.
    Then $G = N H$ if and only if for all $g \in G$ there exists a unique $n \in N$ and h \in H such that $g = nh$.
    %  \begin{enumerate}
    %   \item $G = NH$
    %   \item $N \cap H = e$
    %   \item $N \triangleleft K$
    %  \end{enumerate}

\heading{Wreath products:}

Let $X$ be a finite set, $G$ a group and $H$ a group acting on $X$.
Fix a labelling of $X$ say $\{x_1,x_2, \dots, x_t\}$ with $| X | = t$.
Let $G^t$ be the direct product of $G$ with itself $t$ times.
Then the \textbf{wreath product} of $G$ and $H$ is $G \wr H = G \rtimes H$ where $H$ acts on $G$ by its action on $X$.

% The wreath products of two groups $G$ and $H$ is constructed by: 
% \begin{enumerate}
% \item  Write $H$ as a permutation on $n$ items

% \item Make $n$ copies of the group $G$. $(G^n)$

% \item $H$ acts on the copies of $G$ by permuting the elements.
% \end{enumerate}

% The wreath product of $G$ by $H$ is a semi-direct product of a direct products of $n$ copies
% of $G$ by $H$.

% The wreath product permutes the factors of G according to the action h on X. So if $x \in G$, then the wreath product would take components of G and shuffle them around according to the action h on the set X.
    

\heading{Position Vectors:}
For a given configuration of a Rubik's Cube there exists a $4$ tuple $(\rho, \sigma, v, w))$ where $\rho \in S_8, \sigma \in S_{12}$ describe the permutations of the cubies and $v \in \mathbb Z_3^8, w \in \mathbb Z_2^{12}$ describe the orientations of the cubies.

\textbf{Remark:} 
We will adopt Singmaster notation where  
we label the face with respect to it lying
flat on a plane and one is facing front face)

\begin{itemize}
\item Let $U$ denote the upward (top) face.
\item Let $F$ denote the front face.
\item Let $L$ denote the left face.
\item Let $R$ denote the right face.
\item Let $B$ denote the back face.
\item Let $D$ denote the downward (bottom) face
\end{itemize}

% The capital letter represent a $90\degree$ turn clockwise facing that face. The inverse of each move would be the $90\degree$ rotation of the face counter-clockwise (denoted by a $M^{-1}$ in our adaptation where $M$ \in \{F,L,U,D,R,B \}$).


    % This block catches your eye, so \textbf{important stuff} should probably go
    % here.

    % Curabitur eu libero vehicula, cursus est fringilla, luctus est. Morbi
    % consectetur mauris quam, at finibus elit auctor ac. Aliquam erat volutpat.
    % Aenean at nisl ut ex ullamcorper eleifend et eu augue. Aenean quis velit
    % tristique odio convallis ultrices a ac odio.

    % \begin{itemize}
    %   \item \textbf{Fusce dapibus tellus} vel tellus semper finibus. In
    %     consequat, nibh sed mattis luctus, augue diam fermentum lectus.
    %   \item \textbf{In euismod erat metus} non ex. Vestibulum luctus augue in
    %     mi condimentum, at sollicitudin lorem viverra.
    %   \item \textbf{Suspendisse vulputate} mauris vel placerat consectetur.
    %     Mauris semper, purus ac hendrerit molestie, elit mi dignissim odio, in
    %     suscipit felis sapien vel ex.
    % \end{itemize}

    % Aenean tincidunt risus eros, at gravida lorem sagittis vel. Vestibulum ante
    % ipsum primis in faucibus orci luctus et ultrices posuere cubilia Curae.

  \end{alertblock}
 
  \begin{exampleblock}{Preliminary Theorems}

    \heading{Theorem on parity:} Permutations can also be described in terms of their parity. Any length $n$
cycle of a permutation can be expressed as the product of $2$-cycles.

The cube always has even parity, or an even number of cubies
exchanged from the starting position.

      \heading{Second Fundamental Theorem of Cubology: }
      An operation of the cube is possible if and only if the following are satisfied:
            \begin{itemize}
              \item The total number of edge and corner cycles of even length is even.
              \item The number of corner cycles twisted right is equal to the number of corner cycles twisted left (up to modulo $3$).
              \item There is an even number of reorienting edge cycles.


    % $$
    % \int_{-\infty}^{\infty} e^{-x^2}\,dx = \sqrt{\pi}
    % $$

    % Interdum et malesuada fames $\{1, 4, 9, \ldots\}$ ac ante ipsum primis in
    % faucibus. Cras eleifend dolor eu nulla suscipit suscipit. Sed lobortis non
    % felis id vulputate.

    % \heading{Example of a semi direct product:}

    % The group $S_n$ can be written as a semi-direct product: $S_n =A_n \times \langle$

    % \heading{Another heading inside a block}

    % Sed augue erat, scelerisque a purus ultricies, placerat porttitor neque.
    % Donec $P(y \mid x)$ fermentum consectetur $\nabla_x P(y \mid x)$ sapien
    % sagittis egestas. Duis eget leo euismod nunc viverra imperdiet nec id
    % justo.

    \end{itemize}

  \end{exampleblock}
  

  \begin{block}{Constructing the Legal Rubik's Cube}
    Recall that the Rubik's cube consists of $54$ facets and the entire set of arrangements of the rubiks cube can be thought of as permutations of these $54$ facets. 
    We denote all possible permutations on these facets by $G = \langle F,L,U,D,R,B\rangle \subset S_{54}$ and call it the Rubik's Cube group.

    % \heading{Construction of the illegal Rubik's cube:}

    % First we construct the illegal Rubik's cube group since the legal Rubik's cube group is a subgroup of the illegal Rubik's cube group.
    
    Unlike the legal Rubik's Cube Group, the Illegal Rubik’s Cube Group allows the solver to take the cube apart and rearrange the facets.  
    We also notice that some orientations are theoretically possible but can not be physically realized on the cube. 
    % We also notice that the Rubik's cube group is a subgroup of the illegal rubik's cube group.

    From the description of the corner and edge cubes, we can see that the illegal rubik's cube group is $I = (\mathbb Z^{12}_2 \wr S_{12}) \times (\mathbb Z^8_3 \wr S_8)$.

    % \heading{Construction of the legal Rubik's cube}

    % Now, we will construct the legal Rubik's cube.

    

% Now note that not all arrangments of the permutations would lead to the valid arrangements of a rubik's cube, this group includes all cubes that can be taken apart and put together (the illegal rubiks' cube group), e.g. pieces can alternate between being either corner and edge cubies, and those that can be reached by applying moves on the cube (the legal rubik's cube group).

The corner facets can be represented as the cyclic group of 3 elements $C_{3}$ or $\mathbb Z_3$ and since there are $8$ of these we denote it as the cross product of $C_{3}$ eight times, namely, $C_3 ^8$ 
Now, the possible arrangements of the corner cubes can be described using $S_{8}$ (since we are permuting those), which allows the postions of all the corner facets on the Rubik's Cube to be described as the group $C_{3}^{8} \wr S_{8}$.
Using a similar argument we see that the  position of all the edge facets on the Rubik's Cube can be described by the group $C_{2}^{2} \wr S_{12}$.

  \end{block}

\end{column}

\begin{column}{\colwidth}

  \begin{alertblock}{The Fundamental Theorem of Cubology}
    
    A move sequence is possible if and only if: 
    % the following three conditions are satisfied, we also state the group theoretic formulation:

    \begin{itemize}
      \item \textbf{The permutation of the corner cubies has the same parity as the permutation of the edge cubies} 
      % if and only if 
      $$\text{sign}(\rho) = \text{sign}(\sigma)$$
      \item \textbf{The number of corners that are twisted clockwise is equal to the number that are twisted counterclockwise modulo $3$}
      % (meaning remaining corners twisted in the same direction occur in threes).
      % if and only if 
      $$v_{1} + v_{2} +\dots +v_{8} \equiv 0 \mod 3$$
      \item \textbf{The number of flipped edges is even}
      % if and only if 
      $$w_{1} + w_{2} +\dots +w_{12} \equiv 0 \mod 2$$
    \end{itemize}

  %  \textbf{Remark:}  We are considering the following defintion of a postion of a facet of a cube:
  %   A position vector $(\rho,\sigma,v,w) \in S_{8} \times S_{12} \times C_{3}^{8} \times C_{2}^{12}$

  \end{alertblock}

  \begin{block}{Solvable and Unsolvable Cubes}

    These cubes are unsolvable since they involve odd permutations.

    \begin{enumerate}
      
      \item \textbf{Exactly one edge cubie is flipped}
      A singular edge cubie flipped 90 degrees is not possible since the cube always has even parity.
      
      \item \textbf{Corner twist}
      A singular twisted corner is an impossible case to get on $3 \times 3$.
      As previously stated, the cube always has even parity. However, a corner twist is an odd permutation.
      

Similar to the corner twist, swapped pieces produces odd permutations.
      \item \textbf{Swapped pieces}
      Two edge pieces cannot be swapped, either adjacently or opposite on a $3 \times 3$ Rubik's cube. The same can be said for two corner pieces.
    \end{enumerate}

    \tdplotsetmaincoords{55}{135}
    \begin{figure}
      \centering
            \begin{tikzpicture}[z={(3.85mm,3.85mm)}]
                \coordinate (O) at (0,0,0);
                \coordinate (A) at (1,0,0);
                \coordinate (B) at (2,0,0);
                \coordinate (C) at (3,0,0);
                \coordinate (D) at (0,1,0);
                \coordinate (G) at (1,1,0);
                \coordinate (L) at (2,1,0);
                \coordinate (N) at (3,1,0);
                \coordinate (E) at (0,2,0);
                \coordinate (J) at (1,2,0);
                \coordinate (H) at (2,2,0);
                \coordinate (P) at (3,2,0);
                \coordinate (F) at (0,3,0);
                \coordinate (K) at (1,3,0);
                \coordinate (M) at (2,3,0);
                \coordinate (I) at (3,3,0);
                \coordinate (Q) at (3,0,1);
                \coordinate (R) at (3,0,2);
                \coordinate (S) at (3,0,3);
                \coordinate (T) at (3,1,1);
                \coordinate (U) at (3,1,2);
                \coordinate (V) at (3,1,3);
                \coordinate (W) at (3,2,1);
                \coordinate (X) at (3,2,2);
                \coordinate (Y) at (3,2,3);
                \coordinate (Z) at (3,3,1);
                \coordinate (AA) at (3,3,2);
                \coordinate (BB) at (3,3,3);
                \coordinate (CC) at (0,3,1);
                \coordinate (DD) at (1,3,1);
                \coordinate (EE) at (2,3,1);
                \coordinate (FF) at (0,3,2);
                \coordinate (GG) at (1,3,2);
                \coordinate (HH) at (2,3,2);
                \coordinate (II) at (0,3,3);
                \coordinate (JJ) at (1,3,3);
                \coordinate (KK) at (2,3,3);
                
                \draw[black,fill=blue!80] (O) -- (A) -- (G) -- (D) -- cycle;
                \draw[black,fill=blue!80] (A) -- (B) -- (L) -- (G) -- cycle;
                \draw[black,fill=blue!80] (B) -- (C) -- (N) -- (L) -- cycle;
                \draw[black,fill=blue!80] (D) -- (G) -- (J) -- (E) -- cycle;
                \draw[black,fill=blue!80] (G) -- (L) -- (H) -- (J) -- cycle;
                \draw[black,fill=blue!80] (L) -- (N) -- (P) -- (H) -- cycle;
                \draw[black,fill=blue!80] (E) -- (J) -- (K) -- (F) -- cycle;
                \draw[black,fill=blue!80] (J) -- (H) -- (M) -- (K) -- cycle;
                \draw[black,fill=red!80] (H) -- (P) -- (I) -- (M) -- cycle;
                
                \draw[black,fill=white!80] (C) -- (Q) -- (T) -- (N) -- cycle;
                \draw[black,fill=white!80] (Q) -- (R) -- (U) -- (T) -- cycle;
                \draw[black,fill=white!80] (R) -- (S) -- (V) -- (U) -- cycle;
                \draw[black,fill=white!80] (N) -- (T) -- (W) -- (P) -- cycle;
                \draw[black,fill=white!80] (T) -- (U) -- (X) -- (W) -- cycle;
                \draw[black,fill=white!80] (U) -- (V) -- (Y) -- (X) -- cycle;
                \draw[black,fill=blue!80] (P) -- (W) -- (Z) -- (I) -- cycle;
                \draw[black,fill=white!80] (W) -- (X) -- (AA) -- (Z) -- cycle;
                \draw[black,fill=white!80] (X) -- (Y) -- (BB) -- (AA) -- cycle;
                
                \draw[black,fill=red!80] (F) -- (K) -- (DD) -- (CC) -- cycle;
                \draw[black,fill=red!80] (K) -- (M) -- (EE) -- (DD) -- cycle;
                \draw[black,fill=white!80] (M) -- (I) -- (Z) -- (EE) -- cycle;
                \draw[black,fill=red!80] (CC) -- (DD) -- (GG) -- (FF) -- cycle;
                \draw[black,fill=red!80] (DD) -- (EE) -- (HH) -- (GG) -- cycle;
                \draw[black,fill=red!80] (EE) -- (Z) -- (AA) -- (HH) -- cycle;
                \draw[black,fill=red!80] (FF) -- (GG) -- (JJ) -- (II) -- cycle;
                \draw[black,fill=red!80] (GG) -- (HH) -- (KK) -- (JJ) -- cycle;
                \draw[black,fill=red!80] (HH) -- (AA) -- (BB) -- (KK) -- cycle;
                \end{tikzpicture}
                \caption{Corner twist}
              \end{figure}


              \begin{figure}
                \centering
                \begin{multicols}{4}  
                
                      \begin{tikzpicture}[z={(3.85mm,3.85mm)}]
                          \coordinate (O) at (0,0,0);
                          \coordinate (A) at (1,0,0);
                          \coordinate (B) at (2,0,0);
                          \coordinate (C) at (3,0,0);
                          \coordinate (D) at (0,1,0);
                          \coordinate (G) at (1,1,0);
                          \coordinate (L) at (2,1,0);
                          \coordinate (N) at (3,1,0);
                          \coordinate (E) at (0,2,0);
                          \coordinate (J) at (1,2,0);
                          \coordinate (H) at (2,2,0);
                          \coordinate (P) at (3,2,0);
                          \coordinate (F) at (0,3,0);
                          \coordinate (K) at (1,3,0);
                          \coordinate (M) at (2,3,0);
                          \coordinate (I) at (3,3,0);
                          \coordinate (Q) at (3,0,1);
                          \coordinate (R) at (3,0,2);
                          \coordinate (S) at (3,0,3);
                          \coordinate (T) at (3,1,1);
                          \coordinate (U) at (3,1,2);
                          \coordinate (V) at (3,1,3);
                          \coordinate (W) at (3,2,1);
                          \coordinate (X) at (3,2,2);
                          \coordinate (Y) at (3,2,3);
                          \coordinate (Z) at (3,3,1);
                          \coordinate (AA) at (3,3,2);
                          \coordinate (BB) at (3,3,3);
                          \coordinate (CC) at (0,3,1);
                          \coordinate (DD) at (1,3,1);
                          \coordinate (EE) at (2,3,1);
                          \coordinate (FF) at (0,3,2);
                          \coordinate (GG) at (1,3,2);
                          \coordinate (HH) at (2,3,2);
                          \coordinate (II) at (0,3,3);
                          \coordinate (JJ) at (1,3,3);
                          \coordinate (KK) at (2,3,3);
                          
                          \draw[black,fill=blue!80] (O) -- (A) -- (G) -- (D) -- cycle;
                          \draw[black,fill=blue!80] (A) -- (B) -- (L) -- (G) -- cycle;
                          \draw[black,fill=blue!80] (B) -- (C) -- (N) -- (L) -- cycle;
                          \draw[black,fill=blue!80] (D) -- (G) -- (J) -- (E) -- cycle;
                          \draw[black,fill=blue!80] (G) -- (L) -- (H) -- (J) -- cycle;
                          \draw[black,fill=blue!80] (L) -- (N) -- (P) -- (H) -- cycle;
                          \draw[black,fill=blue!80] (E) -- (J) -- (K) -- (F) -- cycle;
                          \draw[black,fill=green!80] (J) -- (H) -- (M) -- (K) -- cycle;
                          \draw[black,fill=blue!80] (H) -- (P) -- (I) -- (M) -- cycle;
                          
                          \draw[black,fill=white!80] (C) -- (Q) -- (T) -- (N) -- cycle;
                          \draw[black,fill=white!80] (Q) -- (R) -- (U) -- (T) -- cycle;
                          \draw[black,fill=white!80] (R) -- (S) -- (V) -- (U) -- cycle;
                          \draw[black,fill=white!80] (N) -- (T) -- (W) -- (P) -- cycle;
                          \draw[black,fill=white!80] (T) -- (U) -- (X) -- (W) -- cycle;
                          \draw[black,fill=white!80] (U) -- (V) -- (Y) -- (X) -- cycle;
                          \draw[black,fill=white!80] (P) -- (W) -- (Z) -- (I) -- cycle;
                          \draw[black,fill=white!80] (W) -- (X) -- (AA) -- (Z) -- cycle;
                          \draw[black,fill=white!80] (X) -- (Y) -- (BB) -- (AA) -- cycle;
                          
                          \draw[black,fill=red!80] (F) -- (K) -- (DD) -- (CC) -- cycle;
                          \draw[black,fill=red!80] (K) -- (M) -- (EE) -- (DD) -- cycle;
                          \coordinate (T) at (3,1,1);
                          \coordinate (U) at (3,1,2);
                          \coordinate (V) at (3,1,3);
                          \coordinate (W) at (3,2,1);
                          \coordinate (X) at (3,2,2);
                          \coordinate (Y) at (3,2,3);
                          \coordinate (Z) at (3,3,1);
                          \coordinate (AA) at (3,3,2);
                            \draw[black,fill=red!80] (M) -- (I) -- (Z) -- (EE) -- cycle;
                          \draw[black,fill=red!80] (CC) -- (DD) -- (GG) -- (FF) -- cycle;
                          \draw[black,fill=red!80] (DD) -- (EE) -- (HH) -- (GG) -- cycle;
                          \draw[black,fill=red!80] (EE) -- (Z) -- (AA) -- (HH) -- cycle;
                          \draw[black,fill=red!80] (FF) -- (GG) -- (JJ) -- (II) -- cycle;
                          \draw[black,fill=red!80] (GG) -- (HH) -- (KK) -- (JJ) -- cycle;
                          \draw[black,fill=red!80] (HH) -- (AA) -- (BB) -- (KK) -- cycle;
                          \end{tikzpicture}
                          % \end{multicols}
                          % \caption{Swapped pieces}
                        % \end{figure}
          
                        % \begin{figure}
                          \centering
                          % \begin{multicols}{2}
                        
                                \begin{tikzpicture}[z={(3.85mm,3.85mm)}]
                                    \coordinate (O) at (0,0,0);
                                    \coordinate (A) at (1,0,0);
                                    \coordinate (B) at (2,0,0);
                                    \coordinate (C) at (3,0,0);
                                    \coordinate (D) at (0,1,0);
                                    \coordinate (G) at (1,1,0);
                                    \coordinate (L) at (2,1,0);
                                    \coordinate (N) at (3,1,0);
                                    \coordinate (E) at (0,2,0);
                                    \coordinate (J) at (1,2,0);
                                    \coordinate (H) at (2,2,0);
                                    \coordinate (P) at (3,2,0);
                                    \coordinate (F) at (0,3,0);
                                    \coordinate (K) at (1,3,0);
                                    \coordinate (M) at (2,3,0);
                                    \coordinate (I) at (3,3,0);
                                    \coordinate (Q) at (3,0,1);
                                    \coordinate (R) at (3,0,2);
                                    \coordinate (S) at (3,0,3);
                                    \coordinate (T) at (3,1,1);
                                    \coordinate (U) at (3,1,2);
                                    \coordinate (V) at (3,1,3);
                                    \coordinate (W) at (3,2,1);
                                    \coordinate (X) at (3,2,2);
                                    \coordinate (Y) at (3,2,3);
                                    \coordinate (Z) at (3,3,1);
                                    \coordinate (AA) at (3,3,2);
                                    \coordinate (BB) at (3,3,3);
                                    \coordinate (CC) at (0,3,1);
                                    \coordinate (DD) at (1,3,1);
                                    \coordinate (EE) at (2,3,1);
                                    \coordinate (FF) at (0,3,2);
                                    \coordinate (GG) at (1,3,2);
                                    \coordinate (HH) at (2,3,2);
                                    \coordinate (II) at (0,3,3);
                                    \coordinate (JJ) at (1,3,3);
                                    \coordinate (KK) at (2,3,3);
                                    
                                    \draw[black,fill=blue!80] (O) -- (A) -- (G) -- (D) -- cycle;
                                    \draw[black,fill=blue!80] (A) -- (B) -- (L) -- (G) -- cycle;
                                    \draw[black,fill=blue!80] (B) -- (C) -- (N) -- (L) -- cycle;
                                    \draw[black,fill=blue!80] (D) -- (G) -- (J) -- (E) -- cycle;
                                    \draw[black,fill=blue!80] (G) -- (L) -- (H) -- (J) -- cycle;
                                    \draw[black,fill=blue!80] (L) -- (N) -- (P) -- (H) -- cycle;
                                    \draw[black,fill=blue!80] (E) -- (J) -- (K) -- (F) -- cycle;
                                    \draw[black,fill=red!80] (J) -- (H) -- (M) -- (K) -- cycle;
                                    \draw[black,fill=blue!80] (H) -- (P) -- (I) -- (M) -- cycle;
                                    
                                    \draw[black,fill=white!80] (C) -- (Q) -- (T) -- (N) -- cycle;
                                    \draw[black,fill=white!80] (Q) -- (R) -- (U) -- (T) -- cycle;
                                    \draw[black,fill=white!80] (R) -- (S) -- (V) -- (U) -- cycle;
                                    \draw[black,fill=white!80] (N) -- (T) -- (W) -- (P) -- cycle;
                                    \draw[black,fill=white!80] (T) -- (U) -- (X) -- (W) -- cycle;
                                    \draw[black,fill=white!80] (U) -- (V) -- (Y) -- (X) -- cycle;
                                    \draw[black,fill=white!80] (P) -- (W) -- (Z) -- (I) -- cycle;
                                    \draw[black,fill=blue!80] (W) -- (X) -- (AA) -- (Z) -- cycle;
                                    \draw[black,fill=white!80] (X) -- (Y) -- (BB) -- (AA) -- cycle;
                                    
                                    \draw[black,fill=red!80] (F) -- (K) -- (DD) -- (CC) -- cycle;
                                    \draw[black,fill=red!80] (K) -- (M) -- (EE) -- (DD) -- cycle;
                                    \draw[black,fill=red!80] (M) -- (I) -- (Z) -- (EE) -- cycle;
                                    \draw[black,fill=red!80] (CC) -- (DD) -- (GG) -- (FF) -- cycle;
                                    \draw[black,fill=red!80] (DD) -- (EE) -- (HH) -- (GG) -- cycle;
                                    \draw[black,fill=red!80] (EE) -- (Z) -- (AA) -- (HH) -- cycle;
                                    \draw[black,fill=red!80] (FF) -- (GG) -- (JJ) -- (II) -- cycle;
                                    \draw[black,fill=red!80] (GG) -- (HH) -- (KK) -- (JJ) -- cycle;
                                    \draw[black,fill=red!80] (HH) -- (AA) -- (BB) -- (KK) -- cycle;
                                    \end{tikzpicture}
                                    % \end{multicols}
                                    % \caption{Swapped pieces}
                                  % \end{figure}
                                  % \begin{figure}
                                    \centering
                                    % \begin{multicols}{4}
                                          \begin{tikzpicture}[z={(3.85mm,3.85mm)}]
                                              \coordinate (O) at (0,0,0);
                                              \coordinate (A) at (1,0,0);
                                              \coordinate (B) at (2,0,0);
                                              \coordinate (C) at (3,0,0);
                                              \coordinate (D) at (0,1,0);
                                              \coordinate (G) at (1,1,0);
                                              \coordinate (L) at (2,1,0);
                                              \coordinate (N) at (3,1,0);
                                              \coordinate (E) at (0,2,0);
                                              \coordinate (J) at (1,2,0);
                                              \coordinate (H) at (2,2,0);
                                              \coordinate (P) at (3,2,0);
                                              \coordinate (F) at (0,3,0);
                                              \coordinate (K) at (1,3,0);
                                              \coordinate (M) at (2,3,0);
                                              \coordinate (I) at (3,3,0);
                                              \coordinate (Q) at (3,0,1);
                                              \coordinate (R) at (3,0,2);
                                              \coordinate (S) at (3,0,3);
                                              \coordinate (T) at (3,1,1);
                                              \coordinate (U) at (3,1,2);
                                              \coordinate (V) at (3,1,3);
                                              \coordinate (W) at (3,2,1);
                                              \coordinate (X) at (3,2,2);
                                              \coordinate (Y) at (3,2,3);
                                              \coordinate (Z) at (3,3,1);
                                              \coordinate (AA) at (3,3,2);
                                              \coordinate (BB) at (3,3,3);
                                              \coordinate (CC) at (0,3,1);
                                              \coordinate (DD) at (1,3,1);
                                              \coordinate (EE) at (2,3,1);
                                              \coordinate (FF) at (0,3,2);
                                              \coordinate (GG) at (1,3,2);
                                              \coordinate (HH) at (2,3,2);
                                              \coordinate (II) at (0,3,3);
                                              \coordinate (JJ) at (1,3,3);
                                              \coordinate (KK) at (2,3,3);
                                              
                                              \draw[black,fill=blue!80] (O) -- (A) -- (G) -- (D) -- cycle;
                                              \coordinate (T) at (3,1,1);
                                              \coordinate (U) at (3,1,2);
                                              \coordinate (V) at (3,1,3);
                                              \coordinate (W) at (3,2,1);
                                              \coordinate (X) at (3,2,2);
                                              \coordinate (Y) at (3,2,3);
                                              \coordinate (Z) at (3,3,1);
                                              \coordinate (AA) at (3,3,2);
                                              \draw[black,fill=blue!80] (A) -- (B) -- (L) -- (G) -- cycle;
                                              \draw[black,fill=blue!80] (B) -- (C) -- (N) -- (L) -- cycle;
                                              \draw[black,fill=blue!80] (D) -- (G) -- (J) -- (E) -- cycle;
                                              \draw[black,fill=blue!80] (G) -- (L) -- (H) -- (J) -- cycle;
                                              \draw[black,fill=blue!80] (L) -- (N) -- (P) -- (H) -- cycle;
                                              \draw[black,fill=blue!80] (E) -- (J) -- (K) -- (F) -- cycle;
                                              \draw[black,fill=blue!80] (J) -- (H) -- (M) -- (K) -- cycle;
                                              \draw[black,fill=white!80] (H) -- (P) -- (I) -- (M) -- cycle;
                                              
                                              \draw[black,fill=white!80] (C) -- (Q) -- (T) -- (N) -- cycle;
                                              \draw[black,fill=white!80] (Q) -- (R) -- (U) -- (T) -- cycle;
                                              \draw[black,fill=white!80] (R) -- (S) -- (V) -- (U) -- cycle;
                                              \draw[black,fill=white!80] (N) -- (T) -- (W) -- (P) -- cycle;
                                              \draw[black,fill=white!80] (T) -- (U) -- (X) -- (W) -- cycle;
                                              \draw[black,fill=white!80] (U) -- (V) -- (Y) -- (X) -- cycle;
                                              \draw[black,fill=green!80] (P) -- (W) -- (Z) -- (I) -- cycle;
                                              \draw[black,fill=white!80] (W) -- (X) -- (AA) -- (Z) -- cycle;
                                              \draw[black,fill=blue!80] (X) -- (Y) -- (BB) -- (AA) -- cycle;
                                              
                                              \draw[black,fill=red!80] (F) -- (K) -- (DD) -- (CC) -- cycle;
                                              \draw[black,fill=red!80] (K) -- (M) -- (EE) -- (DD) -- cycle;
                                              \draw[black,fill=red!80] (M) -- (I) -- (Z) -- (EE) -- cycle;
                                              \draw[black,fill=red!80] (CC) -- (DD) -- (GG) -- (FF) -- cycle;
                                              \draw[black,fill=red!80] (DD) -- (EE) -- (HH) -- (GG) -- cycle;
                                              \draw[black,fill=red!80] (EE) -- (Z) -- (AA) -- (HH) -- cycle;
                                              \draw[black,fill=red!80] (FF) -- (GG) -- (JJ) -- (II) -- cycle;
                                              \draw[black,fill=red!80] (GG) -- (HH) -- (KK) -- (JJ) -- cycle;
                                              \draw[black,fill=red!80] (HH) -- (AA) -- (BB) -- (KK) -- cycle;
                                              \end{tikzpicture}
                                            % \end{multicols}
                                            % \caption{Swapped pieces}
                                            % \end{figure}

                                            % \begin{figure}
                                              \centering
                                            % \begin{multicols}{4}
                                              
                                                    \begin{tikzpicture}[z={(3.85mm,3.85mm)}]
                                                        \coordinate (O) at (0,0,0);
                                                        \coordinate (A) at (1,0,0);
                                                        \coordinate (B) at (2,0,0);
                                                        \coordinate (C) at (3,0,0);
                                                        \coordinate (D) at (0,1,0);
                                                        \coordinate (G) at (1,1,0);
                                                        \coordinate (L) at (2,1,0);
                                                        \coordinate (N) at (3,1,0);
                                                        \coordinate (E) at (0,2,0);
                                                        \coordinate (J) at (1,2,0);
                                                        \coordinate (H) at (2,2,0);
                                                        \coordinate (P) at (3,2,0);
                                                        \coordinate (F) at (0,3,0);
                                                        \coordinate (K) at (1,3,0);
                                                        \coordinate (M) at (2,3,0);
                                                        \coordinate (I) at (3,3,0);
                                                        \coordinate (Q) at (3,0,1);
                                                        \coordinate (R) at (3,0,2);
                                                        \coordinate (S) at (3,0,3);
                                                        \coordinate (T) at (3,1,1);
                                                        \coordinate (U) at (3,1,2);
                                                        \coordinate (V) at (3,1,3);
                                                        \coordinate (W) at (3,2,1);
                                                        \coordinate (X) at (3,2,2);
                                                        \coordinate (Y) at (3,2,3);
                                                        \coordinate (Z) at (3,3,1);
                                                        \coordinate (AA) at (3,3,2);
                                                        \coordinate (BB) at (3,3,3);
                                                        \coordinate (CC) at (0,3,1);
                                                        \coordinate (DD) at (1,3,1);
                                                        \coordinate (EE) at (2,3,1);
                                                        \coordinate (FF) at (0,3,2);
                                                        \coordinate (GG) at (1,3,2);
                                                        \coordinate (HH) at (2,3,2);
                                                        \coordinate (II) at (0,3,3);
                                                        \coordinate (JJ) at (1,3,3);
                                                        \coordinate (KK) at (2,3,3);
                                                        
                                                        \draw[black,fill=blue!80] (O) -- (A) -- (G) -- (D) -- cycle;
                                                        \draw[black,fill=blue!80] (A) -- (B) -- (L) -- (G) -- cycle;
                                                        \draw[black,fill=blue!80] (B) -- (C) -- (N) -- (L) -- cycle;
                                                        \draw[black,fill=blue!80] (D) -- (G) -- (J) -- (E) -- cycle;
                                                        \draw[black,fill=blue!80] (G) -- (L) -- (H) -- (J) -- cycle;
                                                        \draw[black,fill=blue!80] (L) -- (N) -- (P) -- (H) -- cycle;
                                                        \draw[black,fill=blue!80] (E) -- (J) -- (K) -- (F) -- cycle;
                                                        \draw[black,fill=blue!80] (J) -- (H) -- (M) -- (K) -- cycle;
                                                        \draw[black,fill=green!80] (H) -- (P) -- (I) -- (M) -- cycle;
                                                        
                                                        \draw[black,fill=white!80] (C) -- (Q) -- (T) -- (N) -- cycle;
                                                        \draw[black,fill=white!80] (Q) -- (R) -- (U) -- (T) -- cycle;
                                                        \draw[black,fill=white!80] (R) -- (S) -- (V) -- (U) -- cycle;
                                                        \draw[black,fill=white!80] (N) -- (T) -- (W) -- (P) -- cycle;
                                                        \draw[black,fill=white!80] (T) -- (U) -- (X) -- (W) -- cycle;
                                                        \draw[black,fill=white!80] (U) -- (V) -- (Y) -- (X) -- cycle;
                                                        \draw[black,fill=orange!80] (P) -- (W) -- (Z) -- (I) -- cycle;
                                                        \draw[black,fill=white!80] (W) -- (X) -- (AA) -- (Z) -- cycle;
                                                        \draw[black,fill=white!80] (X) -- (Y) -- (BB) -- (AA) -- cycle;
                                                        
                                                        \draw[black,fill=red!80] (F) -- (K) -- (DD) -- (CC) -- cycle;
                                                        \draw[black,fill=red!80] (K) -- (M) -- (EE) -- (DD) -- cycle;
                                                        \draw[black,fill=red!80] (M) -- (I) -- (Z) -- (EE) -- cycle;
                                                        \draw[black,fill=red!80] (CC) -- (DD) -- (GG) -- (FF) -- cycle;
                                                        \draw[black,fill=red!80] (DD) -- (EE) -- (HH) -- (GG) -- cycle;
                                                        \draw[black,fill=red!80] (EE) -- (Z) -- (AA) -- (HH) -- cycle;
                                                        \draw[black,fill=red!80] (FF) -- (GG) -- (JJ) -- (II) -- cycle;
                                                        \draw[black,fill=red!80] (GG) -- (HH) -- (KK) -- (JJ) -- cycle;
                                                        \draw[black,fill=red!80] (HH) -- (AA) -- (BB) -- (KK) -- cycle;
                                                        \end{tikzpicture}
                                                        
                                                      \end{multicols}
                                                      \caption{Swapped pieces}
                                                      \end{figure}
                                                                                

    % \begin{figure}
    %   \centering
    %   \includegraphics{Screenshot from 2024-03-26 11-14-42.png}
    %   \caption{One edge cubie is flipped}
    % \end{figure}

  \end{block}

  % \begin{block}{An algorithm for solving the Rubik's Cube}

  %   Nulla eget sem quam. Ut aliquam volutpat nisi vestibulum convallis. Nunc a
  %   lectus et eros facilisis hendrerit eu non urna. Interdum et malesuada fames
  %   ac ante \textit{ipsum primis} in faucibus. Etiam sit amet velit eget sem
  %   euismod tristique. Praesent enim erat, porta vel mattis sed, pharetra sed
  %   ipsum. Morbi commodo condimentum massa, \textit{tempus venenatis} massa
  %   hendrerit quis. Maecenas sed porta est. Praesent mollis interdum lectus,
  %   sit amet sollicitudin risus tincidunt non.

  %   Etiam sit amet tempus lorem, aliquet condimentum velit. Donec et nibh
  %   consequat, sagittis ex eget, dictum orci. Etiam quis semper ante. Ut eu
  %   mauris purus. Proin nec consectetur ligula. Mauris pretium molestie
  %   ullamcorper. Integer nisi neque, aliquet et odio non, sagittis porta justo.

  %   \begin{itemize}
  %     \item \textbf{Sed consequat} id ante vel efficitur. Praesent congue massa
  %       sed est scelerisque, elementum mollis augue iaculis.
  %       \begin{itemize}
  %         \item In sed est finibus, vulputate
  %           nunc gravida, pulvinar lorem. In maximus nunc dolor, sed auctor eros
  %           porttitor quis.
  %         \item Fusce ornare dignissim nisi. Nam sit amet risus vel lacus
  %           tempor tincidunt eu a arcu.
  %         \item Donec rhoncus vestibulum erat, quis aliquam leo
  %           gravida egestas.
  %       \end{itemize}
  %     \item \textbf{Sed luctus, elit sit amet} dictum maximus, diam dolor
  %       faucibus purus, sed lobortis justo erat id turpis.
  %     \item \textbf{Pellentesque facilisis dolor in leo} bibendum congue.
  %       Maecenas congue finibus justo, vitae eleifend urna facilisis at.
  %   \end{itemize}

  % \end{block}

\end{column}

\separatorcolumn

\begin{column}{\colwidth}

  \begin{block}{Proof of the Fundamental Theorem of Cubology (Summarized)}

    % $$
    % \int_{-\infty}^{\infty} e^{-x^2}\,dx = \sqrt{\pi}
    % $$

    % Interdum et malesuada fames $\{1, 4, 9, \ldots\}$ ac ante ipsum primis in
    % faucibus. Cras eleifend dolor eu nulla suscipit suscipit. Sed lobortis non
    % felis id vulputate.

    

    

  \end{block}

  \begin{alertblock}{How does this relate to MATH 4GR3?}

    \heading{Commutator}

    % \begin{table}
    %   \centering
    %   \begin{tabular}{l r r c}
    %     \toprule
    %     \textbf{First column} & \textbf{Second column} & \textbf{Third column} & \textbf{Fourth} \\
    %     \midrule
    %     Foo & 13.37 & 384,394 & $\alpha$ \\
    %     Bar & 2.17 & 1,392 & $\beta$ \\
    %     Baz & 3.14 & 83,742 & $\delta$ \\
    %     Qux & 7.59 & 974 & $\gamma$ \\
    %     \bottomrule
    %   \end{tabular}
    %   \caption{A table caption.}
    % \end{table}

    Donec quis posuere ligula. Nunc feugiat elit a mi malesuada consequat. Sed
    imperdiet augue ac nibh aliquet tristique. Aenean eu tortor vulputate,

  \end{alertblock}

  \begin{block}{References}

    \nocite{*}
    \footnotesize{\bibliographystyle{plain}\bibliography{poster}}

  \end{block}

\end{column}

\separatorcolumn
\end{columns}
\end{frame}

\end{document}
